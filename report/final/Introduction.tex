\section{Introduction}

A real efficient lighting solution must be able to accomodate different lighting requirements in different areas, must take into account external lighting in each area and must be able to understand the light interferences between different areas.

The Distributed Lighting Problem consists of finding the optimal way to control a set of illuminaries to keep the illuminance at a desired value. These illuminaires may correspond to different desks or areas which migh have different lighting requirements. The system must present optimal behaviour not only in terms of energy efficiency but also in terms of response speed to changes and confort to the final user. On top of all this a distributed solution when compared to a centralized one presents many benefits. This is the problem we prupose to solve.

There has already been research on this problem \cite{caicedo2013distributed}, \cite{caicedo2011occupancy}, \cite{pandharipande2011daylight}. Our approach tries to define the problem as a linear programm that would then be solved with the Simplex Algorithm. In \cite{DistributedSimplex} a distributed version of this algorithm is explored. In our solution we don't use this version of the simplex algorithm but this might be an interesting improvement.

We created a physical setup to simulate an office environment and test our approach to this problem. It consists of three illuminaires inside a box which could have external lighting. Each illuminaire is made of a LED and a small circuit with an LDR to measure the illuminance, the illuminaire is controlled by an Arduino UNO board. The arduinos communicate between themselves and with a server, this serves as the interface to clients, allowing them to analyze and change the state of the system.

All our code is available publicly on \url{https://github.com/goncalor/SCDTR}.
