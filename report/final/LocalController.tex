% From first report:

%Os seguintes pontos descrevem o que é implementado no código actual:
%Controlador com componentes proporcional, derivativa e integral
%Anti-windup
%Feedforward
%Derivador à saída
%Temporização do controlador através de interrupções
%Dar referências em lux
%Medir iluminância em lux
%Alterar parâmetros do controlador
%Alterar tempo de amostragem
%Implementamos um filtro passa-baixo para o derivador para diminuir o ruído. Não chegámos a afinar a constante desse filtro, pelo que o temos comentado.
%Os parâmetros do controlador foram encontrados com afinação manual e são os seguintes:
%Ganho proporcional: 10
%Ganho diferencial: 0.01
%Ganho integral: 10
%Ganho do feedforward: 0.1
%Ganho do anti-windup: 0.05
%Tempo de amostragem: 1500 µs
%Todos os cálculos do controlador são feitos com valores de 0 a 1023, sendo convertidos quando necessário: para 0 a 255 para actuar o LED; e para lux quando são pedidos valores de iluminância.

\subsection{Local Controller}
\label{sec:LocalController}

The local controller consists of a Proportional-Integral-Derivative Controller (PID) that receives as input a integer reference value (\emph{r}) from 0 to 1023, the LDR readings from 0 to 1023 and has as output the PWM duty cycle (\emph{u}) in the range 0 to 255.

The block diagram of the local controller is on Figure~\ref{fig:pid_block_diagramm}. The error(\emph{e}) is defined as $e = r - u$.

\subsubsection{Proportional Component}
\label{sub:ProportionalComponent}

The proportional component of the controller is just a gain multiplied (\emph{$K_p$}) by \emph{e}.

\subsubsection{Integral Component}
\label{sub:IntegralComponent}

\subsubsection{Derivative Component}
\label{sub:Derivative Component}

\subsubsection{Anti-Windup}
\label{sub:AntiWindup}

\subsubsection{Feedforward}
\label{sub:Feedforward}

\subsubsection{Implementation Details}
\label{sub:Implementation Details}

\subsubsection{Parameters}
\label{sub:Parameters}
