\section{Conclusion}
\label{conclusion}

The final results of this project constitute a control system that allows to control ambient light to meet desired levels specified according to the occupation of parts of a room. The system includes a server that can be connected to a network of luminaires allowing to control the references of each as well as acquired data to measure the energetic efficiency and comfort metrics of the illumination system.

One contribution of this work is the implementation of an interrupt based PID controller that runs on a common microcontroller. Another important contribution is the implementation of the Simplex algorithm according to the pseudocode presented in \cite{Cormen}. This implementation includes the needed transformations for cases where the initial solution is unfeasible. No previous implementation that follows this work was found to exist.

The use of the results provided by the simplex failed to improve the system. Even when the results of the simplex algorithm are applied the system evolves naturally to the state it was before. That state satisfies the illumination requirements but is proved not to be optimal energy-wise.

There is room to improve several aspects of the system. Namely the protocols used could be more robust; the ATMega ADC noise reduction mode could be used to acquire better samples; and the ADC readings could be made in the background by using the interrupts that exist for that purpose and a circular buffer instead of using \texttt{analogRead()} several times and make a mean of the values to filter some noise.



 %
 %Conclusions summarize	the main    achievements    
 %of  the	work.
 %• Also	identify    other   aspects that    are	relevant    for	
 %other	researchers.
 %• Stress    the	contributions	and how	the world   will    
 %be  a	much	better	place	to  live    after   your    work.
 %• Mention   a	few points  worth   looking at	in  future  
 %work,	as  a	consequence of	the conclusions.
